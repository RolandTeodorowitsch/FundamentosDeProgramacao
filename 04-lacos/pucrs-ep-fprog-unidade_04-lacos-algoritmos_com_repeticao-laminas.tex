\documentclass[xcolor={dvipsnames,table},aspectratio=169]{beamer}
\usepackage[utf8]{inputenc}
\usepackage[T1]{fontenc}
\usepackage[brazil]{babel}
\usepackage{graphics,amssymb,amsfonts,amsmath}
\usepackage{tikz}
\usepackage{enumerate,hyperref}
\usepackage{palatino}
\usepackage{ragged2e}
\usepackage{minted}
\usepackage{booktabs}
\usepackage{verbatim}
\usepackage[export]{adjustbox}
\usepackage{tikz}                   
\usepackage{xcolor}
\usepackage{textcomp} % para usar \textdegree
\usepackage{fancyvrb}
\usetikzlibrary{shadows}
\usetheme{AnnArbor}
\usecolortheme{orchid}
\usefonttheme[onlymath]{serif}
\newcommand\setItemnumber[1]{\setcounter{enumi}{\numexpr#1-1\relax}}

\newminted{java}{bgcolor=cyan!10}

\newcolumntype{C}[1]{>{\centering\let\newline\\\arraybackslash\hspace{0pt}}m{#1}}

\AtBeginSection[]{
  \begin{frame}
  \vfill
  \centering
  \begin{beamercolorbox}[sep=8pt,center,shadow=true,rounded=true]{title}
    \usebeamerfont{title}\insertsectionhead\par%
  \end{beamercolorbox}
  \vfill
  \end{frame}
}

\title[\sc{Algoritmos com Repetição}]{Algoritmos com Repetição}
\author[Roland Teodorowitsch]{Roland Teodorowitsch}
\institute[FPROG - EP - PUCRS]{Fundamentos de Programação - Escola Politécnica - PUCRS}
\date{24 de agosto de 2022}

\begin{document}
\justifying

%-------------------------------------------------------
\begin{frame}
	\titlepage
\end{frame}

%=======================================================
\section{Introdução}

%-------------------------------------------------------
\begin{frame}\frametitle{Obra Consultada}

\noindent{ORTH, Afonso Inácio. \textbf{Algoritmos e Programação com Resumo das Linguagens PASCAL e C}. Porto Alegre: AIO, 2001. 176 p.}

\end{frame}

%-------------------------------------------------------
\begin{frame}\frametitle{Conteúdos}
\begin{itemize}
	\item Definições
	\item Variáveis Especiais
	\item Comandos de Repetição
	\item Repetições Aninhadas
	\item Exemplos
	\item Exercícios
\end{itemize}
\end{frame}

%=======================================================
\section{Definições}

%-------------------------------------------------------
\begin{frame}[fragile]\frametitle{Algoritmos com Repetição}
\begin{itemize}
	\item São algoritmos que repetem duas ou mais vezes um ou mais de seus passos
	\item Permitem executar um mesmo trecho de algoritmo sobre diferentes conjuntos de entradas
	\item As construções que permitem a repetição são chamadas \textbf{laços}
\end{itemize}
\end{frame}

%-------------------------------------------------------
\begin{frame}[fragile]\frametitle{O número de repetições pode ser}
\begin{itemize}
	\item Constante (fixo)
    \begin{itemize}
        \item \emph{Calcular a média das notas de uma turma de 10 alunos. (+)}
    \end{itemize}
    \item Fornecido pelo usuário
    \begin{itemize}
        \item \emph{Ler o número de alunos da turma, e calcular a média de suas notas. (+)}
    \end{itemize}
    \item Determinado por condição ou cálculo (\textbf{teste de valor final})
    \begin{itemize}
        \item \emph{Ler as notas dos alunos de uma turma, até que uma nota negativa                 tenha sido fornecida (fim da entrada), calculando a média da turma. (+)}
        \item \emph{Ler as notas dos alunos de uma turma, perguntando após a leitura de cada nota se há mais alguma nota a ser lida, e, no final, calculando a média da turma. (+)}
    \end{itemize}
\end{itemize}
\end{frame}

%=======================================================
\section{Variáveis Especiais}

%-------------------------------------------------------
\begin{frame}[fragile]\frametitle{Variáveis Especiais}
\begin{itemize}
	\item \textbf{Contadora}
	\begin{itemize}
		\item Recebe um valor inicial (geralmente 0) e é incrementada em algum ponto do algoritmo, de um valor CONSTANTE (geralmente 1)
		\item Exemplo\\\texttt{cont = cont + 1}
		\item Pode ser usada para contar eventos ou ocorrências
	\end{itemize}
	\item \textbf{Acumuladora}
	\begin{itemize}
		\item Recebe um valor inicial (geralmente 0) e é incrementada em algum ponto do algoritmo, de um valor VARIÁVEL
		\item Exemplo\\\texttt{soma = soma + variavel}
		\item É usada, por exemplo, para realizar somatórios
	\end{itemize}
	\item \textbf{de Indução}
	\begin{itemize}
		\item É uma variável que controla o número de vezes que um laço irá executar
	\end{itemize}
\end{itemize}
\end{frame}

%=======================================================
\section{Comandos de Repetição}

%-------------------------------------------------------
\begin{frame}[fragile]\frametitle{Tipos de Comandos de Repetição}
\begin{itemize}
	\item Pré-teste
	\begin{itemize}
		\item Primeiro testam, depois executam
		\item O laço pode se repetir nenhuma ou mais vezes
		\item Exemplos\\
		\begin{itemize}
			\item \texttt{Enquanto <condição> fazer ...}
			\item \texttt{Para <variável> = <exp1> até <exp2> [passo <exp3>] fazer ...}
		\end{itemize}
	\end{itemize}
	\item Pós-teste
	\begin{itemize}
		\item Primeiro executam, depois testam
		\item O laço se repete no mínimo uma vez
		\item Exemplos
		\begin{itemize}
			\item \texttt{Fazer ... enquanto <condição>}
			\item \texttt{Repetir ... até <condição>}
		\end{itemize}
	\end{itemize}
\end{itemize}
\end{frame}

%-------------------------------------------------------
\begin{frame}[fragile]\frametitle{Enquanto <condição> fazer ...}
\begin{itemize}
	\item Repete o(s) passo(s) enquanto a condição for verdadeira
	\item Faz o teste antes de executar os passos
	\item É mais geral e flexível
	\item Sempre pode ser usado
\end{itemize}
\end{frame}

%-------------------------------------------------------
\begin{frame}[fragile]\frametitle{Para <variável> = <exp1> até <exp2> [passo <exp3>] fazer ...}
\begin{itemize}
	\item Repete o(s) passo(s) para cada valor da variável
	\item A variação da variável de indução costuma ser regular
	\item Nem sempre é recomendável
	\item Se o passo for positivo, verifica se a variável ainda é menor ou igual a <exp2>
	\item Se o passo for negativo, verifica se a variável ainda é maior ou igual a <exp2>
\end{itemize}
\end{frame}

%-------------------------------------------------------
\begin{frame}[fragile]\frametitle{Fazer ... enquanto <condição>}
\begin{itemize}
	\item Executa determinado(s) passo(s) enquanto a condição for verdadeira
	\item Só sai do laço quando a condição for falsa
	\item C, C++ e Java usam uma forma equivalente
\end{itemize}
\end{frame}

%-------------------------------------------------------
\begin{frame}[fragile]\frametitle{Repetir ... até <condição>}
\begin{itemize}
	\item Executa determinado(s) passo(s) até que a condição seja verdadeira
	\item Só sai do laço quando a condição for verdadeira
	\item Pascal usa uma forma equivalente
\end{itemize}
\end{frame}

%-------------------------------------------------------
\newbox\verbboxA
\newbox\verbboxB
\newbox\verbboxC
\newbox\verbboxD
\setbox\verbboxA=\vbox{\hsize=2.75in\footnotesize
\begin{Verbatim}
i=1
Enquanto i<=5 fazer Início
  Escrever(i)
  i=i+1
Fim
\end{Verbatim}
}
\setbox\verbboxB=\vbox{\hsize=2.75in\footnotesize
\begin{Verbatim}
Para i=1 até 5 fazer
  Escrever(i)



\end{Verbatim}
}
\setbox\verbboxC=\vbox{\hsize=2.75in\footnotesize
\begin{Verbatim}
i=1
Fazer Início
  Escrever(i)
  i=i+1
Fim
enquanto i<=5
\end{Verbatim}
}
\setbox\verbboxD=\vbox{\hsize=2.75in\footnotesize
\begin{Verbatim}
i=1
Repetir Início
  Escrever(i)
  i=i+1
Fim
até i>5
\end{Verbatim}
}
\begin{frame}[fragile]\frametitle{Exemplo 1 (1,2,3,4,5)}
\begin{center}
\begin{tabular}{|l|l|}\hline
\box\verbboxA & \box\verbboxB\\\hline
\box\verbboxC & \box\verbboxD\\\hline
\end{tabular}
\end{center}
\end{frame}

%-------------------------------------------------------
\newbox\verbboxA
\newbox\verbboxB
\newbox\verbboxC
\newbox\verbboxD
\setbox\verbboxA=\vbox{\hsize=2.75in\footnotesize
\begin{Verbatim}
i=2
Enquanto i<=10 fazer Início
  Escrever(i)
  i=i+2
Fim
\end{Verbatim}
}
\setbox\verbboxB=\vbox{\hsize=2.75in\footnotesize
\begin{Verbatim}
Para i=2 até 10 passo 2 fazer
  Escrever(i)



\end{Verbatim}
}
\setbox\verbboxC=\vbox{\hsize=2.75in\footnotesize
\begin{Verbatim}
i=2
Fazer Início
  Escrever(i)
  i=i+2
Fim
enquanto i<=10
\end{Verbatim}
}
\setbox\verbboxD=\vbox{\hsize=2.75in\footnotesize
\begin{Verbatim}
i=2
Repetir Início
  Escrever(i)
  i=i+2
Fim
até i>10
\end{Verbatim}
}
\begin{frame}[fragile]\frametitle{Exemplo 2 (2,4,6,8,10)}
\begin{center}
\begin{tabular}{|l|l|}\hline
\box\verbboxA & \box\verbboxB\\\hline
\box\verbboxC & \box\verbboxD\\\hline
\end{tabular}
\end{center}
\end{frame}

%-------------------------------------------------------
\newbox\verbboxA
\newbox\verbboxB
\newbox\verbboxC
\newbox\verbboxD
\setbox\verbboxA=\vbox{\hsize=2.75in\footnotesize
\begin{Verbatim}
i=1.0
Enquanto i>=0.0 fazer Início
  Escrever(i)
  i=i-0.25
Fim  
\end{Verbatim}
}
\setbox\verbboxB=\vbox{\hsize=2.75in\footnotesize
\begin{Verbatim}
Para i=1.0 até 0.0 passo -0.25 fazer
  Escrever(i)



\end{Verbatim}
}
\setbox\verbboxC=\vbox{\hsize=2.75in\footnotesize
\begin{Verbatim}
i=1.0
Fazer Início
  Escrever(i)
  i=i-0.25
Fim
enquanto >=0.0
\end{Verbatim}
}
\setbox\verbboxD=\vbox{\hsize=2.75in\footnotesize
\begin{Verbatim}
i=1.0
Repetir Início
  Escrever(i)
  i=i-0.25
Fim
até i<0.0
\end{Verbatim}
}
\begin{frame}[fragile]\frametitle{Exemplo 3 (1.0,0.75,0.5,0.25,0.0)}
\begin{center}
\begin{tabular}{|l|l|}\hline
\box\verbboxA & \box\verbboxB\\\hline
\box\verbboxC & \box\verbboxD\\\hline
\end{tabular}
\end{center}
\end{frame}

%-------------------------------------------------------
\newbox\verbboxA
\newbox\verbboxB
\newbox\verbboxC
\newbox\verbboxD
\setbox\verbboxA=\vbox{\hsize=2.75in\footnotesize
\begin{Verbatim}
i=1
Enquanto i<=16 fazer Início
  Escrever(i)
  i=i*2
Fim   
\end{Verbatim}
}
\setbox\verbboxC=\vbox{\hsize=2.75in\footnotesize
\begin{Verbatim}
i=1
Fazer Início
  Escrever(i)
  i=i*2
Fim
enquanto i<=16
\end{Verbatim}
}
\setbox\verbboxD=\vbox{\hsize=2.75in\footnotesize
\begin{Verbatim}
i=1
Repetir Início
  Escrever(i)
  i=i*2
Fim
até i>16
\end{Verbatim}
}
\begin{frame}[fragile]\frametitle{Exemplo 4 (1,2,4,8,16)}
\begin{center}
\begin{tabular}{|l|l|}\hline
\box\verbboxA & ~\\\hline
\box\verbboxC & \box\verbboxD\\\hline
\end{tabular}
\end{center}
\end{frame}

%=======================================================
\section{Repetições Aninhadas}

%-------------------------------------------------------
\begin{frame}[fragile]\frametitle{Repetições Aninhadas}
\begin{itemize}
	\item São repetições dentro de outras repetições
	\item É possível ter vários níveis de laços aninhados
	\item Exemplos
	\begin{itemize}
		\item Um único laço sem aninhamento
		\begin{itemize}
            \item \emph{Verificar se um número lido é primo. (+)}
            \item \emph{Calcular o fatorial de um número. (+)}
            \item \emph{Ler um vetor (array de 1 dimensão) (-)}
        \end{itemize}
		\item Dois laços aninhados
		\begin{itemize}
            \item \emph{Ler 10 números, verificando se eles são primos. (+)}
            \item \emph{Ler valores e, enquanto nenhum valor negativo for fornecido, calcular o fatorial de cada valor. (+)}
            \item \emph{Ler uma matriz (array de 2 dimensões). (-)}
        \end{itemize}
		\item Três laços aninhados
		\begin{itemize}
            \item \emph{Ler um array tridimensional. (-)}
            \item \emph{Multiplicar 2 matrizes. (-)}
        \end{itemize}
	\end{itemize}
\end{itemize}
\end{frame}

%=======================================================
\section{Exemplos}

%-------------------------------------------------------
\newbox\verbboxA
\setbox\verbboxA=\vbox{\hsize=3in\footnotesize
\begin{Verbatim}
Algoritmo MediaNota10Alunos
  cont: inteiro
  nota, media: Reais
Início
  cont = 0
  media = 0.0
  Enquanto cont < 10 fazer Início
    Ler(nota)
    media = media + nota
    cont = cont + 1
  Fim
  media = media / cont
  Escrever(media)
Fim
\end{Verbatim}
}
\begin{frame}[fragile]\frametitle{Exemplo 1}
\small{\emph{Calcular a média das notas de uma turma de 10 alunos.}}
\begin{center}
\begin{tabular}{|l|}\hline
\box\verbboxA \\\hline
\end{tabular}
\end{center}
\end{frame}

%-------------------------------------------------------
\newbox\verbboxA
\setbox\verbboxA=\vbox{\hsize=3in\scriptsize
\begin{Verbatim}
Algoritmo MediaNotaNAlunos
  cont, n: inteiro
  nota, media: Reais
Início
  cont = 0
  media = 0.0
  Ler(n)
  Enquanto cont < n fazer Início
    Ler(nota)
    media = media + nota
    cont = cont + 1
  Fim
  Se cont>0 então Início
    media = media / cont
    Escrever(media)
  Fim
Fim
\end{Verbatim}
}
\begin{frame}[fragile]\frametitle{Exemplo 2}
\small{\emph{Ler o número de alunos da turma, e calcular a média de suas notas.}}
\begin{center}
\begin{tabular}{|l|}\hline
\box\verbboxA \\\hline
\end{tabular}
\end{center}
\end{frame}

%-------------------------------------------------------
\newbox\verbboxA
\setbox\verbboxA=\vbox{\hsize=3in\tiny
\begin{Verbatim}
Algoritmo MediaNotaNAlunosV2
  cont, n: inteiro
  nota, media: Reais
Início
  cont = 0
  media = 0.0
  Ler(nota)
  Enquanto nota>=0 fazer Início
    media = media + nota
    cont = cont + 1
    Ler(nota)
  Fim
  Se cont>0 então Início
    media = media / cont
    Escrever(media)
  Fim
Fim
\end{Verbatim}
}
\begin{frame}[fragile]\frametitle{Exemplo 3}
\small{\emph{Ler as notas dos alunos de uma turma, até que uma nota negativa tenha sido fornecida (fim da entrada), calculando a média da turma.}}
\begin{center}
\begin{tabular}{|l|}\hline
\box\verbboxA \\\hline
\end{tabular}
\end{center}
\end{frame}

%-------------------------------------------------------
\newbox\verbboxA
\setbox\verbboxA=\vbox{\hsize=3in\tiny
\begin{Verbatim}
Algoritmo MediaNotaNAlunosV3
  cont, n: inteiro
  nota, media: Reais
  continuar: Texto
Início
  cont = 0
  media = 0.0
  Ler(nota)
  Ler(continuar)
  Enquanto continuar="S" fazer Início
    media = media + nota
    cont = cont + 1
    Ler(nota)
    Ler(continuar)
  Fim
  Se cont>0 então Início
    media = media / cont
    Escrever(media)
  Fim
Fim
\end{Verbatim}
}
\begin{frame}[fragile]\frametitle{Exemplo 4}
\small{\emph{Ler as notas dos alunos de uma turma, perguntando após a leitura de cada nota se há mais alguma nota a ser lida, e, no final, calculando a média da turma.}}
\begin{center}
\begin{tabular}{|l|}\hline
\box\verbboxA \\\hline
\end{tabular}
\end{center}
\end{frame}

%-------------------------------------------------------
\newbox\verbboxA
\setbox\verbboxA=\vbox{\hsize=4.5in
\begin{Verbatim}
Algoritmo EhPrimo
  div, n, i: inteiro
Início
  Ler(n)
  div = 0
  Para i=1 até n passo 1 fazer Início
    Se n % i == 0 então div = div + 1
  Fim
  Se div == 2 então Escrever("Eh primo")
              senão Escrever("NAO eh primo")
Fim
\end{Verbatim}
}
\begin{frame}[fragile]\frametitle{Exemplo 5}
\small{\emph{Verificar se um número lido é primo.}}
\begin{center}
\begin{tabular}{|l|}\hline
\box\verbboxA \\\hline
\end{tabular}
\end{center}
\end{frame}

%-------------------------------------------------------
\newbox\verbboxA
\setbox\verbboxA=\vbox{\hsize=4.5in\small
\begin{Verbatim}
Algoritmo QuaisDos10SaoPrimos
  div, n, i, cont: inteiro
Início
  Para cont=1 até 10 passo 1 fazer Início
    Ler(n)
    div = 0
    Para i=1 até n passo 1 fazer Início
      Se n % i == 0 então div = div + 1
    Fim
    Se div == 2 então Escrever("Eh primo")
                senão Escrever("NAO eh primo")
  Fim
Fim
\end{Verbatim}
}
\begin{frame}[fragile]\frametitle{Exemplo 6}
\small{\emph{Ler 10 números, verificando se eles são primos.}}
\begin{center}
\begin{tabular}{|l|}\hline
\box\verbboxA \\\hline
\end{tabular}
\end{center}
\end{frame}

%-------------------------------------------------------
\newbox\verbboxA
\setbox\verbboxA=\vbox{\hsize=4in
\begin{Verbatim}
Algoritmo Fatorial
  fat, n, i: inteiro
Início
  Ler(n)
  fat = 1
  Para i=2 até n passo 1 fazer Início
    fat = fat * i
  Fim
  Escrever(fat)
Fim
\end{Verbatim}
}
\begin{frame}[fragile]\frametitle{Exemplo 7}
\small{\emph{Calcular o fatorial de um número.}}
\begin{center}
\begin{tabular}{|l|}\hline
\box\verbboxA \\\hline
\end{tabular}
\end{center}
\end{frame}

%-------------------------------------------------------
\newbox\verbboxA
\setbox\verbboxA=\vbox{\hsize=4in\small
\begin{Verbatim}
Algoritmo FatorialDeVariosNumeros
  fat, n, i: inteiro
Início
  Ler(n)
  Enquanto n>=0 fazer Início
    fat = 1
    Para i=2 até n passo 1 fazer Início
      fat = fat * i
    Fim
    Escrever(fat)
    Ler(n)
  Fim
Fim
\end{Verbatim}
}
\begin{frame}[fragile]\frametitle{Exemplo 8}
\small{\emph{Ler valores e, enquanto nenhum valor negativo for fornecido, calcular o fatorial de cada valor.}}
\begin{center}
\begin{tabular}{|l|}\hline
\box\verbboxA \\\hline
\end{tabular}
\end{center}
\end{frame}

%=======================================================
\section{Exercícios}

%-------------------------------------------------------
\begin{frame}[fragile]\frametitle{Exercícios 1-5}
\begin{enumerate}[1.]
{\small
\item Escrever um algoritmo que lê 5 valores reais para \texttt{a}, um de cada vez, e conta quantos destes valores são negativos, escrevendo esta informação.
\item Escrever um algoritmo que gera e escreve os números ímpares entre 100 e 200.
\item Escrever um algoritmo que lê 10 valores reais, um de cada vez, e conta quantos deles estão no intervalo [10,20] e quantos deles estão fora deste intervalo, escrevendo estas informações.
\item Escrever um algoritmo que lê um número não conhecido de valores reais, um de cada vez, e conta quantos deles estão em cada um dos intervalos [0,25), [25,50), [50,75) e [75,100], escrevendo estas informações. A leitura deve ser feita até que um valor fora dos intervalos seja fornecido.
\item Escrever um algoritmo semelhante ao anterior, que calcula as médias aritméticas de cada intervalo e as escreve, juntamente com o número de valores encontrados em cada intervalo. Caso nenhum valor tenha sido encontrado em determinado intervalo, escrever ``*'' no lugar da média.
}
\end{enumerate}
\end{frame}

%-------------------------------------------------------
\begin{frame}[fragile]\frametitle{Exercícios 6-8}
\begin{enumerate}[1.]
\setItemnumber{6}
{\small
\item A série de Fibonacci tem como dados os 2 primeiros termos da série que são respectivamente 0 e 1. À partir deles, os demais termos são construídos pela seguinte regra:
\[ t_{n} = t_{n-1} + t_{n-2} \]
Escrever um algoritmo que gera os 10 primeiros termos desta série e calcula e escreve a sua soma.
\item Escrever um algoritmo que gera os 10 primeiros termos da série de Fibonacci, escrevendo para cada termo gerado o número de ordem e o valor do termo da série de Fibonacci. Considere que os números de ordem iniciam com 0.
\item Escrever um algoritmo que gera os 30 primeiros termos da série de Fibonacci e escreve os termos gerados com a mensagem: ``EH PRIMO'' ou ``NAO EH PRIMO'', conforme o caso.
}
\end{enumerate}
\end{frame}

%=======================================================
\end{document}

