\documentclass[onecolumn,a4paper,10pt]{report}
%\documentclass[12pt,a4paper,twoside]{book} %twoside distingue página par de ímpar
\usepackage[utf8]{inputenc}
\usepackage[portuges]{babel} %para separar sílabas em Português, etc...
\usepackage[usenames,dvipsnames]{color} % para letras e caixas coloridas
\usepackage{latexsym} %para fazer $\Box$ no \LaTeX2$\epsilon$
\usepackage{makeidx} % índice remissivo
\usepackage{amstext} %texto em equações: $... \text{} ...$
\usepackage{theorem}
\usepackage{tabularx} %tabelas ocupando toda a página
\usepackage[all]{xy}
\usepackage{a4wide} %correta formatação da página em A4
\usepackage{indentfirst} %adiciona espaços no primeiro parágrafo

\usepackage{graphics,amssymb,amsfonts,amsmath}
\usepackage{tikz}
\usepackage{enumerate,hyperref}
\usepackage{palatino}
\usepackage{ragged2e}
\usepackage{minted}
\usepackage{booktabs}
\usepackage{verbatim}
\usepackage[export]{adjustbox}
\usepackage{tikz}                   
\usepackage{xcolor}
\usepackage{textcomp} % para usar \textdegree
\usepackage{setspace}
\usetikzlibrary{shadows}

\newminted{java}{bgcolor=cyan!10}

\definecolor{cinza}{gray}{.8}
\definecolor{branco}{gray}{1}
\definecolor{preto}{gray}{0}
\definecolor{verdemusgo}{rgb}{.3,.7,.5}
\definecolor{vinho}{cmyk}{0,1,1,.5}
%\setcounter{secnumdepth}{1}
%\renewcommand{\thesection}{\textcolor{preto}{\arabic{section}}}
%\renewcommand{\thepage}{\textcolor{preto}{\color{preto}{{\scriptsize}}}}
{\theorembodyfont{\upshape}
\newtheorem{Dem}{Demonstração}[chapter]}
\newtheorem{Ex}{Exemplo}[chapter]
\newtheorem{Exer}{Exercício}
\newtheorem{Lista}{Lista de exercícios}
\newtheorem{Def}{Definição}[chapter]

\newtheorem{Pro}{Proposição}[chapter]
\newtheorem{Ax}{Axioma}[chapter]
\newtheorem{Teo}{Teorema}[chapter]
\newtheorem{Cor}{Corolário}[chapter]
\newtheorem{Cas}{Caso}[subsection]
\newtheorem{lema}{Lema}[chapter]
\newtheorem{que}{Questão}[chapter]
\newcommand{\dem}{\noindent{\bf Demonstração:}}
\newcommand{\sol}{\noindent{\it Solução.}}
\newcommand{\nota}{\noindent{\bf Notação:}}
\newcommand{\ex}{\noindent{\bf Exemplos}}
\newcommand{\Obs}{\noindent{\bf Observação:}}
\newcommand{\fim}{\hfill $\blacksquare$}
\newcommand{\ig}{\,\, = \,\,}
\newcommand{\+}{\, + \,}
\newcommand{\m}{\, - \,}
\newcommand{\I}{\mbox{$I\kern-0.40emI$}}
\newcommand{\Z}{\mbox{Z$\kern-0.40em$Z}}
\newcommand{\Q}{\mbox{I$\kern-0.60em$Q}}
\newcommand{\C}{\mbox{I$\kern-0.60em$C}}
\newcommand{\N}{\mbox{I$\kern-0.40em$N}}
\newcommand{\R}{\mbox{I$\kern-0.40em$R}}
\newcommand{\Ro}{\rm{I\!R\!}}
\newcommand{\disp}{\displaystyle}
\newcommand{\<}{\hspace*{-0.4cm}}
\newcommand{\ds}{\displaystyle}
\newcommand{\ov}{\overline}
\newcommand{\aj}{\vspace*{-0.2cm}}
\newcommand{\pt}{\hspace{-1mm}\times\hspace{-1mm}}
\newcommand{\cm}{\mbox{cm}}
\newcommand{\np}{\mbox{$\in \kern-0.80em/$}}
\newcommand{\tg} {\mbox{tg\,}}
\newcommand{\ptm}{\hspace{-0.4mm}\cdot\hspace{-0.4mm}}
\newcommand{\arc}{\stackrel{\;\;\frown}}
\newcommand{\rad}{\;\mbox{rad}}
\newcommand{\esp}{\;\;\;\;}
\newcommand{\sen}{\mbox{sen\,}}
\newcommand{\grau}{^{\mbox{{\scriptsize o}}}}
\newcommand{\real} {\mbox{$I\kern-0.60emR$}}
\newcommand{\vetor}{\stackrel{\color{vinho}\vector(1,0){15}}}
\newcommand{\arctg}{\mbox{arctg\,}}
\newcommand{\arcsen}{\mbox{arcsen\,}}
\newcommand{\ordinal}{^{\underline{\scriptsize\mbox{\rmo}}}}
\newcommand{\segundo}{$2^{\underline{o}}$ }
\newcommand{\primeiro}{$1^{\underline{o}}$ }
\newcommand{\nee}{\mbox{$\;=\kern-0.90em/\;$}}

\setlength{\parskip}{0.0cm} %espaco entre parágrafos
\setlength{\oddsidemargin}{-1cm} %margem esquerda das páginas
%\setlength{\unitlength}{3cm} %tamanho da figura criada
\linespread{1.5} %distância entre linhas
\setlength{\textheight}{25cm} %distância entre a primeira e última linha do texto(comprimento do texto)
\setlength{\textwidth}{18cm} %indica a largura do texto
\topmargin=-2cm %margem superior entre topo da página e o cabeçalho
%\headsep=0.5cm %distãncia entre o cabeçalho e o corpo do texto
%\setlength{\footskip}{27pt} %distãncia da última linha ao número da página
%\evensidemargin=-0.2in %margem esquerda das páginas pares
%\marginparwidth=1.7in %tamanho das notas de margem
%\marginparsep=0.2in %distância entre a margem direita e as notas de margem
%\topmargin=0cm
%\stackrel{\frown}{AB}

\begin{document}
\singlespacing

\begin{center}
Pontifícia Universidade Católica do Rio Grande do Sul (PUCRS)\\
Escola Politécnica\\
Disciplina: Fundamentos de Programação - Professor: Roland Teodorowitsch\\
24 de agosto de 2022
\end{center}

\begin{center}
\textbf{Lista de Exercícios - Unidade 3: Decisões}
\end{center}

\begin{enumerate}[1.]

%----------------------------------------------------------------------
\item Qual o valor de cada variável dos trechos de código Java a seguir após o respectivo comando \texttt{if}?
\begin{enumerate}[a)]
\item
\begin{javacode}
int n = 1; int k = 2; int r = n;
if (k < n) { r = k; }
\end{javacode}
\item
\begin{javacode}
int n = 1; int k = 2; int r;
if (n < k) { r = k; }
else { r = k + n; }
\end{javacode}
\item
\begin{javacode}
int n = 1; int k = 2; int r = k;
if (r < k) { n = r; }
else { k = n; }
\end{javacode}
\item
\begin{javacode}
int n = 1; int k = 2; int r = 3;
if (r < n + k) { r = 2 * n; }
else { k = 2 * r; }
\end{javacode}
\end{enumerate}
{\tiny Fonte: Horstmann (2013, p. 121, R3.1)}

%----------------------------------------------------------------------
\item O que está errado em cada um dos fragmentos de código Java a seguir?
\begin{enumerate}[a)]
\item
\begin{javacode}
if x > 0 then System.out.print(x);
\end{javacode}
\item
\begin{javacode}
if (1 + x > Math.pow(x, Math.sqrt(2)) { y = y + x; }
\end{javacode}
\item
\begin{javacode}
if (x = 1) { y++; }
\end{javacode}
\item
\begin{javacode}
x = in.nextInt();
if (in.hasNextInt()) {
   sum = sum + x;
}
else {
   System.out.println("Bad input for x");
}
\end{javacode}
\item
\begin{javacode}
String letterGrade = "F";
if (grade >= 90) { letterGrade = "A"; }
if (grade >= 80) { letterGrade = "B"; }
if (grade >= 70) { letterGrade = "C"; }
if (grade >= 60) { letterGrade = "D"; }
\end{javacode}
\end{enumerate}
{\tiny Fonte: Horstmann (2013, p. 122, R3.3)}

%----------------------------------------------------------------------
\item O que os fragmentos de código Java a seguir imprimem?
\begin{enumerate}[a)]
\item
\begin{javacode}
int n = 1;
int m = -1;
if (n < -m) { System.out.print(n); }
else { System.out.print(m); }
\end{javacode}
\item
\begin{javacode}
int n = 1;
int m = -1;
if (-n >= m) { System.out.print(n); }
else { System.out.print(m); }
\end{javacode}
\item
\begin{javacode}
double x = 0;
double y = 1;
if (Math.abs(x - y) < 1) { System.out.print(x); }
else { System.out.print(y); }
\end{javacode}
\item
\begin{javacode}
double x = Math.sqrt(2);
double y = 2;
if (x * x == y) { System.out.print(x); }
else { System.out.print(y); }
\end{javacode}
\end{enumerate}
{\tiny Fonte: Horstmann (2013, p. 122, R3.4)}

%----------------------------------------------------------------------
\item Escreva um programa em Java que lê um valor inteiro e imprime se ele é negativo, zero ou positivo.\\
{\tiny Fonte: Horstmann (2013, p. 126, P3.1)}

%----------------------------------------------------------------------
\item Escreva um programa em Java que lê um número em ponto-flutuante e imprime: ``zero'', se o número é zero. Caso contrário, imprime ``positivo" ou ``negativo'', conforme o caso. O programa também deve imprimir ``pequeno'', se o valor absoluto do número for menor do que $1$, ou ``grande'', se o número exceder $1.000.000$\\
{\tiny Fonte: Horstmann (2013, p. 126, P3.2)}

%----------------------------------------------------------------------
\item Escreva um programa em Java que lê as coordenadas de um ponto no plano cartesiano, \texttt{x} e \texttt{y},
e determine se este ponto corresponde à origem, se está sobre um dos eixos (eixo $x$ ou eixo $y$) ou em qual
quadrante o ponto se encontra (primeiro, segundo, terceiro ou quarto quadrante).\\
{\tiny Autor: Roland Teodorowitsch}

%----------------------------------------------------------------------
\item Escreva um programa em Java que lê um inteiro e imprime quantos dígitos o número tem, verificando se o número é $\geqslant 10$, $\geqslant 100$, e assim por diante. (Assuma que todos os inteiros sejam menores do que dez bilhões.)
Se o número for negativo, primeiro multiplique ele por $-1$.\\
{\tiny Fonte: Horstmann (2013, p. 126, P3.3)}

%----------------------------------------------------------------------
\item Escreva um programa em Java que lê três números e imprime: "todos iguais", se eles forem todos iguais;
"todos diferentes", se forem todos diferentes; e "nenhuma alternativa", em caso contrário.\\
{\tiny Fonte: Horstmann (2013, p. 126, P3.4)}

%----------------------------------------------------------------------
\item Escreva um programa em Java que lê três números e imprime: ``crescentes'', se eles estiverem em ordem crescente;
``decrescentes'', se eles estiverem em ordem decrescente; e ``nenhuma ordem aparente'', em caso contrário. A ordem deve ser
seguida estritamente, ou seja, cada valor deverá ser maior do que seu predecessor (ou menor, conforme o teste). A sequência
$3 4 4$, por exemplo, não deve ser considerada crescente.\\
{\tiny Fonte: Horstmann (2013, p. 126, P3.5)}

%----------------------------------------------------------------------
\item Escrever um programa em Java que lê 3 valores (\texttt{a}, \texttt{b} e \texttt{c}) e calcula e escreve
a média ponderada com peso 5 para o maior dos 3 valores e peso 2,5 para os outros dois.\\
{\tiny Adaptado de: Orth (2001, p. 39)}

%----------------------------------------------------------------------
\item Escrever um programa em Java que lê dois valores (\texttt{a} e \texttt{b}) e verifica se eles são múltiplos ou não.
Escrever os números e uma mensagem (``são múltiplos'' ou ``não são múltiplos'').\\
{\tiny Adaptado de: Orth (2001, p. 39)}

%----------------------------------------------------------------------
\item Escrever um programa em Java que lê 4 números inteiros e imprime: ``dois pares'', se na entrada houver dois pares
de números iguais (em qualquer posição), ou ``sem dois pares'', em caso contrário. Por exemplo, em ``1 2 2 1'' e ``2 2 2 2'' há ``dois pares''.
Para ``1 2 2 3'', por outro lado, seria impresso ``sem dois pares''.\\
{\tiny Fonte: Horstmann (2013, p. 126, P3.8)}

%----------------------------------------------------------------------
\item Escreva um programa em Java que leia 3 \emph{strings} e ordene elas lexicograficamente. Por exemplo:\\
\texttt{Forneca 3 strings: \textbf{Charlie Alfa Bravo}\\
Alfa\\
Bravo\\
Charlie}\\
{\tiny Fonte: Horstmann (2013, p. 127, P3.16)}

%----------------------------------------------------------------------
\item Quando dois instantes de tempo são comparados, cada um dado em horas (variando de 0 a 23) e minutos, o seguinte pseudocódigo determina qual ocorre antes:
\begin{verbatim}
If hour1 < hour2
   time1 comes first.
Else if hour1 and hour2 are the same
   If minute1 < minute2
      time1 comes first.
   Else if minute1 and minute2 are the same
      time1 and time2 are the same.
   Else
      time2 comes first.
Else
   time2 comes first.
\end{verbatim}
Escreva um programa em Java que solicite dois horários ao usuário (cada hora formada por dois valores inteiros: horas e minutos) e imprima o menor horário antes e o maior depois.\\
{\tiny Fonte: Horstmann (2013, p. 127, P3.17)}

%----------------------------------------------------------------------
\item O seguinte algoritmo descobre a estação do ano (Primavera, Verão, Outono ou Inverno) para determinado mês e dia.
\begin{verbatim}
If month is 1, 2, or 3, season = "Winter"
Else if month is 4, 5, or 6, season = "Spring"
Else if month is 7, 8, or 9, season = "Summer"
Else if month is 10, 11, or 12, season = "Fall"
If month is divisible by 3 and day >= 21
   If season is "Winter", season = "Spring"
   Else if season is "Spring", season = "Summer"
   Else if season is "Summer", season = "Fall"
   Else season = "Winter"
\end{verbatim}
Escreva um programa em Java que solicite o dia e o mês
ao usuário e então descubra e imprima a estação do ano.
O algoritmo acima funciona para o hemisfério em que nos encontramos? Caso não funcione adapte-o.\\
{\tiny Adaptado de: Horstmann (2013, p. 128, P3.18)}

\end{enumerate}

%----------------------------------------------------------------------
\noindent{\textbf{REFERÊNCIAS}}

\begin{comment}
\noindent{FORBELLONE, André Luiz Villar; EBERSPÄCHER, Henri Frederico. \textbf{Lógica de programação}: a construção de algoritmos e estruturas de dados. 3. ed. São Paulo: Prentice Hall, 2005. 218 p.}
\end{comment}

\noindent{HORSTMANN, C. \textbf{Java for Everyone – Late Objetct}. 2. ed. Hoboken: Wiley, 2013. xxxiv, 589 p.}

\noindent{ORTH, Afonso Inácio. \textbf{Algoritmos e Programação com Resumo das Linguagens PASCAL e C}. Porto Alegre: AIO, 2001. 176 p.}

\end{document}

